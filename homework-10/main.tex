\documentclass[12pt]{article}
\usepackage{amsmath,amsthm,amssymb,amsfonts,setspace,setspace,graphicx,natbib,color,xfrac}

\textwidth 7.0 truein
\oddsidemargin -0.25in   %left-hand edge
\evensidemargin -0.5 truein  %right-hand edge
\topmargin -0.85in      %top of paper to top of head, pulls whole unit
\textheight 9.5in

%%%% In most cases you won't need to edit anything above this line %%%%

\begin{document}
\hfill Tanner Kvarfordt

\hfill Math 2270

\hfill Assignment \#10


\begin{itemize}
\item[6.2.2)] If $A$ has $\lambda_1=2$ with eigenvector $\vec{x}_1=\begin{bmatrix}1 \\ 0\end{bmatrix}$ and
		 	  $\lambda_2=5$ with $\vec{x}_2 = \begin{bmatrix}1 \\ 1\end{bmatrix}$, use $X\Lambda X^{-1}$ to find 
              $A$.
              No other matrix has the same $\lambda$'s and $\vec{x}$'s.

\textit{Solution.}
$X=\begin{bmatrix}
1 & 1 \\ 0 & 1
\end{bmatrix}\Rightarrow
X^{-1}=\begin{bmatrix}
1 & -1 \\ 0 & 1
\end{bmatrix}$, and using the eigenvalues, $\Lambda= 
\begin{bmatrix}
2 & 0 \\ 0 & 5
\end{bmatrix}$\\
so $X\Lambda X^{-1}=\begin{bmatrix}
2 & 3 \\ 0 & 5
\end{bmatrix}=A$

\item[6.2.11)] True or false: If the eigenvalues of $A$ are 2, 2, 5 then the matrix is certainly
			   \begin{itemize}
			   \item[a)] invertible
               \item[b)] diagonalizable
               \item[c)] not diagonalizable
			   \end{itemize}

\textit{Solution.}
\begin{itemize}
\item[a)] True. $|A|=(2)(2)(5)=20\neq0$ so $A$ is invertible.
\item[b)] False. When $\lambda=2$, it is possible for the algebraic multiplicity and geometric multiplicity of $\lambda$ to be equivalent, but not certain.
\item[c)] False. See part (b) for explanation.
\end{itemize}

\item[6.2.16)] Find $\Lambda$ and $X$ to diagonalize $A_1$ in Problem 15 (See Textbook). 
			   What is the limit of $\Lambda^k$ as $k \rightarrow\infty$? 
               What is the limit of $X\Lambda^k X^{-1}$? 
               In the columns of this limiting matrix you see the $\rule{1cm}{0.15mm}$.
               \[A_1=\begin{bmatrix}
               \sfrac{3}{5} & \sfrac{9}{10} \\ \sfrac{2}{5} & \sfrac{1}{10}
               \end{bmatrix}\]
            
\textit{Solution.}
\begin{itemize}
\item[a)] $|A_1-\lambda I|=\begin{bmatrix}
			\sfrac{3}{5} -\lambda & \sfrac{9}{10} \\ \sfrac{2}{5} & \sfrac{1}{10} -\lambda
			\end{bmatrix}
            =\lambda^2-\dfrac{7}{10}\lambda-\dfrac{3}{10}=(\lambda-1)(\lambda+\dfrac{3}{10})
            \Rightarrow \lambda=1,\dfrac{-3}{10}$ \\
            
            $(A_1-\lambda_1 I)\vec{x}=\vec{0}\Rightarrow
            \begin{bmatrix}
            \sfrac{-2}{5} & \sfrac{9}{10} \\ \sfrac{2}{5} & \sfrac{-9}{10}
            \end{bmatrix}\vec{x}=\vec{0}\longrightarrow
            \begin{bmatrix}
            \sfrac{-2}{5} & \sfrac{9}{10} \\ 0 & 0
            \end{bmatrix}\vec{x}=\vec{0}$ free var $x_2=1$, \\
            $\Rightarrow x_1=\sfrac{9}{4}$, so $\vec{v}_1=\begin{bmatrix}\sfrac{9}{4} \\ 1\end{bmatrix}$ \\
            
            $(A_1-\lambda_2 I)\vec{x}=\vec{0}\Rightarrow
            \begin{bmatrix}
            \sfrac{9}{10} & \sfrac{9}{10} \\ \sfrac{2}{5} & \sfrac{2}{5}
            \end{bmatrix}\vec{x}=\vec{0}\longrightarrow
            \begin{bmatrix}
            \sfrac{9}{10} & \sfrac{9}{10} \\ 0 & 0
            \end{bmatrix}\vec{x}=\vec{0}$ free var $x_2=1$, \\
            $\Rightarrow x_1=-1$, so $\vec{v}_{\sfrac{-3}{10}}=\begin{bmatrix} -1 \\ 1\end{bmatrix}$ \\
            Therefore 
            $A=X \Lambda X^{-1}=
            \begin{bmatrix}
            \sfrac{9}{4} & -1 \\ 1 & 1
            \end{bmatrix}
            \begin{bmatrix}
            1 & 0 \\ 0 & \sfrac{-3}{10}
            \end{bmatrix} X^{-1}$
\item[b)] $\lim_{k\rightarrow\infty}\Lambda^k=
			\begin{bmatrix}
			1^k & 0 \\ 0 & (\frac{-3}{10})^k
			\end{bmatrix}=
            \begin{bmatrix}
            1 & 0 \\ 0 & 0
            \end{bmatrix}$
\item[c)] $\lim_{k\rightarrow\infty}X\Lambda^k X^{-1}=
 			\begin{bmatrix}
            \sfrac{9}{4} & -1 \\ 1 & 1
            \end{bmatrix}
            \begin{bmatrix}
            1 & 0 \\ 0 & 0
            \end{bmatrix}X^{-1}=
            \begin{bmatrix}
            \sfrac{9}{4} & 0 \\ 1 & 0
            \end{bmatrix}X^{-1}$
\item[d)] the eigenvector for $\lambda=1$
\end{itemize}

\end{itemize}

\noindent \textbf{Problem S1}: I have argued in class that the eigenvectors of a
matrix $\mathbf{A}$ determine the behavior of $\lim_{n\rightarrow \infty} 
\mathbf{A}^n \vec{x}$.  What this means is that the vector 
$\mathbf{A}^{100}\vec{x}$ should look approximately like one of the eigenvectors
of $\mathbf{A}$.  In this problem, we will show this for the symmetric matrix 
$\mathbf{A} =  \left[\begin{array}{cc} 5/4 & -3/4\\ -3/4 & 5/4\end{array}\right]
$. 
\begin{itemize}
\item[(a)] Diagonalize $\mathbf{A}$.\\  (Hint: To make the algebra easier, I 
suggest that you do not use orthonormal vectors in $\mathbf{S}$.)
\item[(b)] Let $\vec{v}_1$ and $\vec{v}_2$ be the two eigenvectors of 
$\mathbf{A}$.  Using the material from section 4.2, compute the projections of 
an arbitrary vector $\vec{x}$ onto the two eigenvectors, i.e., compute 
$proj_{\vec{v}_1}(\vec{x})$ and $proj_{\vec{v}_2}(\vec{x})$. 
\item[(c)] Explain why $\vec{x} = proj_{\vec{v}_1}(\vec{x})+proj_{\vec{v}_2}
(\vec{x})$.
\item[(d)] Using the answers to the above parts, compute 
$\mathbf{A}^{100}\vec{x}$ where $\vec{x} = [2,1]^T$.
\item[(e)] Since $0.5^{100}$ is approximately zero, what is the approximate 
value of $\mathbf{A}^{100}\vec{x}$ for $\vec{x} = [2,1]^T$? How does that 
approximation relate to the eigenvectors of $\mathbf{A}$?
\end{itemize}

\textit{Solution.}
\begin{itemize}
\item[a)] $|A-\lambda I| = 
			\left|\begin{matrix}
            \sfrac{5}{4}-\lambda & \sfrac{-3}{4} \\ \sfrac{3}{4} & \sfrac{5}{4} - \lambda
			\end{matrix}\right|=\lambda^2-\dfrac{5}{2}\lambda+1=(\lambda-\sfrac{1}{2})(\lambda-2)=0
            \Rightarrow \lambda=\dfrac{1}{2},2$ \\
            $(A-\lambda_2I)=\begin{bmatrix}
            \sfrac{-3}{4} & \sfrac{-3}{4} \\ \sfrac{-3}{4} & \sfrac{-3}{4}
            \end{bmatrix} \Rightarrow \begin{bmatrix}
            \sfrac{-3}{4} & \sfrac{-3}{4} \\ \sfrac{-3}{4} & \sfrac{-3}{4}
            \end{bmatrix}\vec{x}=\vec{0}\longrightarrow
            \begin{bmatrix}
            \sfrac{-3}{4} & \sfrac{-3}{4} \\ 0 & 0
            \end{bmatrix}\vec{x}=\vec{0} \Rightarrow$ free var $x_2=1\Rightarrow \vec{v}_2=\begin{bmatrix}
            -1 \\ 1
            \end{bmatrix}\Rightarrow \vec{v}_{\frac{1}{2}}=\begin{bmatrix}
            1 \\ 1
            \end{bmatrix}$ \\
            Therefore $A=S\Lambda S^{-1}=
            \begin{bmatrix}1 & -1 \\ 1 & 1\end{bmatrix} 
            \begin{bmatrix}\sfrac{1}{2} & 0 \\ 0 & 2\end{bmatrix} 
            S^{-1}$
\item[b)] $P=\frac{\vec{v}_1^T\vec{x}}{\vec{v}_1^T\vec{v}_1}\vec{v_1}=\begin{bmatrix}
			\sfrac{1}{2} \\ \sfrac{-1}{2}
			\end{bmatrix}$ \\
            $P=\frac{\vec{v}_2^T\vec{x}}{\vec{v}_2^T\vec{v}_2}\vec{v_2}=\begin{bmatrix}
			\sfrac{3}{2} \\ \sfrac{3}{2}
			\end{bmatrix}$
\item[c)] Because $A$ is symmetric, and because $\vec{v}_1$ is orthogonal to $\vec{v}_2$.
\item[d)] \[A^{100}\vec{x}=S\Lambda^{100} S^{-1}\vec{x}=
			\begin{bmatrix}
			 1 & -1 \\ 1 & 1
			\end{bmatrix}
            \begin{bmatrix}
            (\sfrac{1}{2})^{100} & 0 \\ 0 & 2^{100}
            \end{bmatrix}
            \begin{bmatrix}
            \sfrac{1}{2} & \sfrac{1}{2} \\ -\sfrac{1}{2} & \sfrac{1}{2}
            \end{bmatrix}
            \begin{bmatrix}
            2 \\ 1
            \end{bmatrix}\approx
            \begin{bmatrix}
            0 & -(2^{100}) \\ 0 & 2^{100}
            \end{bmatrix}
            \begin{bmatrix}
            \sfrac{3}{2} \\ -\sfrac{1}{2}
            \end{bmatrix}\approx
            \begin{bmatrix}
            2^{99} \\ -(2^{99})
            \end{bmatrix}\]
\item[e)] See part (c) for approximation of $\mathbf{A}^{100}\vec{x}$. It relates to the eigenvectors of $A$ in that it is a scalar multiple of the eigenvector $\begin{bmatrix}1 \\ -1 \end{bmatrix}$
\end{itemize}

\noindent \textbf{Problem S2}:  The results in problem S1 hold for any $n\times 
n$ matrix that has $n$ linearly independent eigenvectors, i.e., when 
$AM(\lambda_i) = GM(\lambda_i)$ for every eigenvalue.  Let $\vec{v}_1$, ..., 
$\vec{v}_n$ denote the eigenvectors of $\mathbf{A}$ and $\lambda_1$, ..., 
$\lambda_n$ be the corresponding eigenvalues.  
\begin{itemize}
\item[(a)] Explain why the eigenvectors form a basis for $\mathbf{R}^n$.
\end{itemize}    
Because the eigenvectors form a basis for $\mathbf{R}^n$, we can write an 
arbitrary vector $\vec{x}$ in $\mathbf{R}^n$ as a linear combination of the 
eigenvectors: $\vec{x} = c_1\vec{v}_1+...+c_n\vec{v}_n$. 
\begin{itemize}
\item[(b)] Using the linear combination $\vec{x} = c_1\vec{v}_1+...
+c_n\vec{v}_n$, compute $\mathbf{A}^{100}\vec{x}$. 
\item[(c)] If $\lambda_1>1$ and $|\lambda_i|<1$ for $2\leq i\leq n$, what is the
approximate value of $\mathbf{A}^{100}\vec{x}$? 
\end{itemize} 

\textit{Solution.}
\begin{itemize}
\item[a)] The eigenvectors form a basis of $\mathbf{R}^n$ because they are $n$ linearly independent vectors.
\item[b)] $\mathbf{A}^{100}\vec{x}=c_1^{100}\vec{v}_1+c_2^{100}\vec{v}_2+...+c_n^{100}\vec{v}_n$
\item[c)] If all eigenvectors but the first one are less than $1$, they will rapidly approach $0$ as they are raised by several orders of magnitude. Therefore $\mathbf{A}^{100}\vec{x}=c_1^{100}\vec{v_1}$
\end{itemize}

\noindent \textbf{Problem S3}: In this problem you will explore the dynamics of 
a population with two stages: juveniles and adults.  Note this means you will be
using a 2x2 matrix model. 
\begin{itemize}
\item[(a)] Throughout we will assume that juveniles mature after one year and 
cannot reproduce.  Because of this assumption, what entry of the 2x2 matrix must
be zero?
\item[(b)] Assume that adults produce 5 offspring per year, the proportion of 
surviving adults each year is 0.5, and the proportion of juveniles that survive 
to become adults is 0.05.
\begin{itemize}
\item[(i)] Write down the matrix model for this case and determine the 
eigenvalues and eigenvectors for the matrix.
\item[(ii)] Is the population growing or decreasing?
\end{itemize}
\item[(c)] Assume that adults produce 5 offspring per year and the proportion of
surviving adults each year is 0.5. You will now determine what proportion of the
juvenile population needs to mature each year in order for a population to 
increase.
\begin{itemize}
\item[(i)] Let $p$ denote the proportion of juveniles that survive to become 
adults. Create a new matrix model with this unknown proportion.  
\item[(ii)] Compute the eigenvalues for the matrix. (Hint: use the quadratic 
formula.)
\item[(iii)] Using the formula from part (ii), find the value of $p$ that 
results in one eigenvalue being equal to one. 
\end{itemize}
\end{itemize}

\textit{Solution.}
\begin{itemize}
\item[a)] The 1,1 entry must be 0.
\item[b)]
	\begin{itemize}
	\item[i)] $\left|\begin{bmatrix}
			    0 & 5 \\ 0.05 & 0.5	
			   \end{bmatrix} -\lambda I\right|=\lambda^2-0.5\lambda-0.25=
               \left(\lambda+\frac{\sqrt[]{5} - 1}{4} \right)\left(\lambda-\frac{\sqrt[]{5} + 1}{4} \right)=0$ \\
               $\Rightarrow \lambda_1=\frac{-\sqrt[]{5} + 1}{4}$, $\lambda_2=\frac{\sqrt[]{5} + 1}{4}$ \\
               Finding the eigenvector for $\lambda_1$:
               $\begin{bmatrix}
               \frac{\sqrt[]{5}-1}{4} & 5 \\ 0.05 & 0.5-\frac{-\sqrt[]{5}+1}{4}
               \end{bmatrix}\vec{x}=\vec{0} \longrightarrow 
               \begin{bmatrix}
               \frac{\sqrt[]{5}-1}{4} & 5 \\ 0 & 0
               \end{bmatrix}\vec{x}=\vec{0}$ \\
               Since free variable $x_2=1$, $x_1=\frac{-20}{\sqrt[]{5}-1} \Rightarrow 
               \vec{v}_{\lambda_1}= \begin{bmatrix} \frac{-20}{\sqrt[]{5}-1} \\ 1 \end{bmatrix}$ \\
               Finding eigenvector for $\lambda_2$: 
               $\begin{bmatrix}
               \frac{-\sqrt[]{5}-1}{4} & 5 \\ 0.05 & 0.5 - \frac{\sqrt[]{5}+1}{4}
               \end{bmatrix}\vec{x}=\vec{0} \longrightarrow 
                \begin{bmatrix} \frac{-\sqrt[]{5}-1}{4} & 5 \\ 0 & 0 \end{bmatrix}\vec{x}=\vec{0}$ \\
                Since free variable $x_2=1$, $x_1=\frac{-20}{-\sqrt[]{5}-1} \Rightarrow 
                \vec{v}_{\lambda_2}=\begin{bmatrix} \frac{-20}{-\sqrt[]{5}-1} \\ 1\end{bmatrix}$
	\item[ii)] Since neither eigenvalue is greater than one, the population is decreasing.
	\end{itemize}
\item[c)]
	\begin{itemize}
	\item[i)] $\begin{bmatrix}
			   	0 & 5 \\ p & 0.5
			  	\end{bmatrix}$
	\item[ii)] $\left|\begin{matrix}
				-\lambda & 5 \\ p & 0.5-\lambda
				\end{matrix}\right|=\lambda^2-0.5\lambda-5p=0 \rightarrow
                \lambda = \frac{0.5\pm\sqrt[]{0.25-4(-5p)}}{2}=\frac{0.5\pm\sqrt[]{0.25+20p}}{2}$
    \item[iii)] In order for an eigenvalue to be greater than 1, $0.5\pm\sqrt[]{0.25+20p}$ must be
    			greater than 2. In order for that to happen, $\sqrt[]{0.25+20p}$ must be greater than $1.5$.
                $(1.5)^2=2.25$, so $0.25+20p > 2.25 \Rightarrow p > 0.1$ in order for an eigenvector to be greater 
                than 1, and hence for the population to increase.
	\end{itemize}
\end{itemize}

\noindent \textbf{Problem S4} \noindent Consider a population whose annual 4x4 
Leslie matrix is $\mathbf{M} = \left[\begin{array}{cccc} 0 & 1 & 2 & 3\\  0.6 & 
0.1 & 0 & 0\\ 0 & 0.5 & 0.2 & 0\\ 0 & 0 & 0.3 & 0.1 \end{array}\right]$.\\  
Assume the 4 population stages are babies, small adults, medium adults, and 
large adults. 
\begin{itemize}
\item[1)] (a) What is the biological interpretation of the 1,3 entry of 
$\mathbf{M}$? \\
(b) What is the biological interpretation of the 4,4 entry of $\mathbf{M}$?\\
(c) What is the biological interpretation of the 2,1 entry of $\mathbf{M}$?
\item[2)] The eigenvalues of $\mathbf{M}$ are 1.237, -0.55, and $-0.1447 \pm 
0.55 i$.  Will the population experience long term growth or decline?
\item[3)] Imagine that a fraction $p$ of the surviving small adults became large
adults in one year (they skipped the medium adult stage).  What zero entry of 
$\mathbf{M}$ would be replaced by $p$?
\end{itemize}

\textit{Solution.}
\begin{itemize}
\item[1)] (a) Medium adults produce 2 babies per year. \\
		  (b) $M_{4,4}$ is the proportion of surviving large adults. \\
          (c) $M_{2,1}$ is the proportion of babies maturing to small adults.
\item[2)] Since at least one eigenvalue is greater than 1, the population will experience long term growth.
\item[3)] $M_{4,2}$ would be replaced by $p$.
\end{itemize}
\end{document}
$\vec{u}=\left[\begin{array}{c} 1 \\ 0\end{array}\right]$
$\left[\begin{array}{cc}  & \\  & \end{array}\right]$
$\left[\begin{array}{ccc}  &  & \\  &  & \\ & & \end{array}\right]$

\end{document}


CREATE VECTOR
$\vec{u}=\left[\begin{array}{c} 1 \\ 0\end{array}\right]$

CREATE EQUATION
$A = \begin{align}{cc}
 1 & 2\\
3 & 4
\end{align}$

CREATE EQUATION
\begin{equation}
y=x
\end{equation}

CREATE SYSTEM OF EQUATIONS
\begin{align}
y&=x\\
y&=x
\end{equation}

