% pdflatex Homework12.tex; evince Homework12.pdf & 
\documentclass[12pt,pdftex]{article}
\usepackage{amsmath,amsthm,amsfonts,setspace,graphicx,natbib,color,xfrac,gensymb}
\textwidth 7.0 truein
\oddsidemargin -0.25in   %left-hand edge
\evensidemargin -0.5 truein  %right-hand edge
\topmargin -0.85in      %top of paper to top of head, pulls whole unit
\textheight 9.5in

\begin{document}

\hfill Tanner Kvarfordt 

\hfill A02052217

\hfill Math 2270

\hfill Assignment \#12

\begin{itemize}
\item[8.1.7)] For these transformations of $V=R^2$ to $W=R^2$, find $T(T(\vec{v}))$. Show that when $T(\vec{v})$ is linear, then also $T(T(\vec{v}))$ is linear.
\begin{itemize}
\item[a)] $T(\vec{v})=-\vec{v}$
\item[b)] $T(\vec{v})=\vec{v}+(1,1)$
\item[c)] $T(\vec{v})=90\degree$ rotation $=(-v_2, v_1)$
\item[d)] $T(\vec{v})=$projection$=\frac{1}{2}(v_1+v_2, v_1+v_2)$
\end{itemize}

\textit{Solution.}
If a transformation is linear, that property holds regardless of input, so if $T(\vec{v})$ is linear, then 
$T(T(\vec{v}))$ will also be linear.
\begin{itemize}
\item[a)] $T(T(\vec{v}))=\vec{v}$ \\
			$T(c\vec{v}_1+d\vec{v}_2)=-c\vec{v}_1-d\vec{v}_2=cT(\vec{v}_1)+dT(\vec{v}_2)$ shows that $T(\vec{v})$ is linear.
\item[b)] $T(T(\vec{v}))=\vec{v}+(2,2)$ \\
			$T(c\vec{v}_1+d\vec{v}_2)=c\vec{v}_1+d\vec{v}_2+(1,1) \neq
            cT(\vec{v}_1)+dT(\vec{v}_2)=c(\vec{v}_1+(1,1))+d(\vec{v}_2+(1,1))$ shows that $T(\vec{v})$ is not linear.
\item[c)] $T(T(\vec{v}))=(-v_1,-v_2)$ \\
			$T(c\vec{v}+d\vec{w})=\begin{bmatrix}
			-cv_2-dw_2 \\ cv_1+dw_1
			\end{bmatrix} = c\begin{bmatrix}
			-v_2 \\ v_1
			\end{bmatrix}+d\begin{bmatrix}
			-w_2 \\ w_1
			\end{bmatrix}=cT(\vec{v})+dT(\vec{w})$ shows that $T(\vec{v})$ is linear.
\item[d)] $T(T(\vec{v}))=\frac{1}{4}\begin{bmatrix}
			2(v_1+v_2) \\ 2(v_1+v_2)
			\end{bmatrix} = \frac{1}{2}\begin{bmatrix}
			v_1+v_2 \\ v_1+v_2
			\end{bmatrix}$ \\
            $T(c\vec{v}+d\vec{w})=\frac{1}{2}\begin{bmatrix}
            cv_1+dw_1+cv_2+dw_2 \\ cv_1+dw_1+cv_2+dw_2
            \end{bmatrix} = \frac{1}{2}c\begin{bmatrix}
            v_1+v_2 \\ v_1+v_2
            \end{bmatrix}+\frac{1}{2}d\begin{bmatrix}
            w_1+w_2 \\ w_1+w_2
            \end{bmatrix}=cT(\vec{v})+dT(\vec{w})$ shows that $T(\vec{v})$ is linear.
\end{itemize}

\item[8.1.8)] Find the range and kernel (like the column space and nullspace) of $T$:
\begin{itemize}
\item[a)] $T(v_1,v_2)=(v_1-v_2,0)$
\item[b)] $T(v_1,v_2,v_3)=(v_1,v_2)$
\item[c)] $T(v_1,v_2)=(0,0)$
\item[d)] $T(v_1,v_2)=(v_1,v_1)$
\end{itemize}

\textit{Solution.}
\begin{itemize}
\item[a)] range: span$\{(1, 0)\}$\\
			kernel: When $v_1=v_2$, or $\{(x,x)\}$ where $x$ is any real number.
\item[b)] range: $\mathbb{R}^2$\\
			kernel: span$\{(0,0,1)\}$
\item[c)] range: $\{(0,0)\}$\\
			kernel: $\mathbb{R}^2$
\item[d)] range: span$\{(1,1)\}$\\
			kernel: span$\{(0,1)\}$
\end{itemize}

\item[8.2.1)] The transformation $S$ takes the second derivative. Keep $1,x,x^2,x^3$ as the input basis 
				$v_1,v_2,v_3,v_4$ and also as output basis $w_1,w_2,w_3,w_4$. Write $S(v_1),S(v_2),S(v_3),S(v_4)$ in terms
                of the $w$'s. Find the 4 by 4 matrix $A$ for $S$.

\textit{Solution.}
\begin{itemize}
\item[a)] $v_1,v_2,v_3,v_4=w_1,w_2,w_3,w_4=1,x,x^2,x^3$\\
		$\Rightarrow S(v_1)=S(v_2)=0$, $S(v_3)=2w_1=2$, $S(v_4)=6w_2=6x$
\item[b)] Consider vectors in $\mathcal{P}_i$ to be represented in $\mathbb{R}^i$ as $\begin{bmatrix}
			\text{coef on } x^0 \\ \text{coef on } x^1 \\ \vdots \\ \text{coef on } x^{i - 1} \\ \text{coef on } x^i
			\end{bmatrix}$. Therefore, using the the outputs given in part (a), the matrix is $A=\begin{bmatrix}
			0 & 0 & 2 & 0 \\ 0 & 0 & 0 & 6 \\ 0 & 0 & 0 & 0 \\ 0 & 0 & 0 & 0
			\end{bmatrix}$
\end{itemize}

\item[8.2.10)] Suppose $T(v_1)=w_1+w_2+w_3$ and $T(v_2)=w_2+w_3$ and $T(v_3)=w_3$. Find the matrix $A$ for $T$ using these basis vectors. What input vector $v$ gives $T(v)=w_1$?

\textit{Solution.}
\begin{itemize}
\item[a)] Using $\vec{w}$ as the output vector, $A$ can be constructed by using each output as a column. So 
			$A=\begin{bmatrix}
			1 & 0 & 0 \\ 1 & 1 & 0 \\ 1 & 1 & 1
			\end{bmatrix}$.
\item[b)] To get output $(1,0,0)$, $\vec{v}=(1,-1,0)$. $A\vec{v}=(1,0,0)$
\end{itemize}

\item[8.2.26)] Suppose $v_1,v_2,v_3$ are eigenvectors for $T$. This means $T(v_i)=\lambda_iv_i$ for $i=1,2,3$. What is the matrix for $T$ when the input and output bases are the $v$'s?

\textit{Solution.} 
$T(v_i)=\lambda_iv_i \Rightarrow Av_i=\lambda_iv_i \Rightarrow 11A=\Lambda= \begin{bmatrix}
\lambda_1 & 0 & 0 \\ 0 & \lambda_2 & 0 \\ 0 & 0 & \lambda_3
\end{bmatrix}$

\item[7.4.18)] Find $A^+$ and $A^+A$ and $AA^+$ and $x^+$ for this matrix $A=U\Sigma V^T$ and these $b$:
\[A=\begin{bmatrix}
3 \\ 4
\end{bmatrix}=\begin{bmatrix}
0.6 & -0.8 \\ 0.8 & 0.6
\end{bmatrix}\begin{bmatrix}
5 \\ 0
\end{bmatrix}\begin{bmatrix}
1
\end{bmatrix}\]
\[b=\begin{bmatrix}
3 \\ 4
\end{bmatrix} \text{ and } b= \begin{bmatrix}
-4 \\ 3
\end{bmatrix}\]

\textit{Solution.}
\begin{itemize}
\item[a)] $A^+=\begin{bmatrix}
			1
			\end{bmatrix}
			\begin{bmatrix}
			\frac{1}{5} & 0
			\end{bmatrix}\begin{bmatrix}
			\sfrac{3}{5} & \sfrac{4}{5} \\ \sfrac{-4}{5} & \sfrac{3}{5}
			\end{bmatrix}=\begin{bmatrix}
			\sfrac{3}{25} & \sfrac{4}{25}
			\end{bmatrix}$
\item[b)] $A^+A=1$
\item[c)] $AA^+=\begin{bmatrix}
			\sfrac{9}{25} & \sfrac{12}{25} \\ \sfrac{12}{25} & \sfrac{6}{25}
			\end{bmatrix}$
\item[d)] $x^+=A^+\begin{bmatrix}
			3 \\ 4
			\end{bmatrix}=1$
\item[e)] $x^+=A^+\begin{bmatrix}
			-4 \\ 3
			\end{bmatrix}=0$
\end{itemize}

\end{itemize}  

\noindent \textbf{Problem S1}: Consider the matrix $A=\left(\begin{array}{cc} 1 
& 2\\ 3 & 6\end{array}\right)$. 
\begin{itemize}
\vspace{-10pt}
\item[(a)] Compute the pseudoinverse $A^+$.  
\item[(b)] As in Problem 4b from Exam 2, draw the four fundamental subspaces of 
$A$ on two 2D graphs.  Each plot should contain exactly two orthogonal 
subspaces. Be sure to label each subspace.
\item[(c)] On the appropriate graph from part (b), draw a vector $\vec{b}$ that 
is a linear combination of vectors in $\mathcal{C}(A^T)$ and $\mathcal{N}(A)$.  
Draw a second vector that shows which subspace $A\vec{b}$ is in.
\item[(d)] On the appropriate graph from part (b), draw a vector $\vec{c}$ that 
is a linear combination of vectors in $\mathcal{C}(A)$ and $\mathcal{N}(A^T)$.  
Draw a second vector that shows which subspace $A^+\vec{c}$ is in.\\
\end{itemize}

\textit{Solution.}
\begin{itemize}
\item[a)] $|A^TA-\lambda I|=\lambda(\lambda-50)=0\Rightarrow\lambda=0,50$ \\
			$(A^TA-50\lambda)\vec{x}=\vec{0}\longrightarrow \begin{bmatrix}
			-40 & 20 \\ 0 & 0
			\end{bmatrix}\vec{x}=\vec{0}$ free var $x_2=1 \\ 
            \Rightarrow x_1=\frac{1}{2} \Rightarrow \vec{w}_1=(\frac{1}{2}, 1)
            \Rightarrow\vec{v}_1=\begin{bmatrix}\sfrac{1}{\sqrt[]{5}} \\ \sfrac{2}{\sqrt[]{5}}\end{bmatrix}$ \\
            Since $V$ has orthonormal columns, 
            infer $\vec{v}_2=\begin{bmatrix}
            \frac{-2}{\sqrt[]{5}} \\ \frac{1}{\sqrt[]{5}}
            \end{bmatrix}$ \\
            $A\vec{v}_1=\sigma_1\vec{u}_1=\begin{bmatrix}
            \sfrac{5}{\sqrt[]{5}} \\ \sfrac{15}{\sqrt[]{5}}
            \end{bmatrix}=\sqrt[]{50}\begin{bmatrix}
            \frac{1}{\sqrt[]{10}} \\ \frac{3}{\sqrt[]{10}}
            \end{bmatrix}$ Since $U$ has orthonormal columns, 
            infer $\vec{u}_2=\begin{bmatrix}
            \frac{-3}{\sqrt[]{10}} \\ \frac{1}{\sqrt[]{10}}
            \end{bmatrix}$ \\
            Therefore $A=U\Sigma V^T=\begin{bmatrix}
            \frac{1}{\sqrt[]{10}} & \frac{-3}{\sqrt[]{10}} \\ \frac{3}{\sqrt[]{10}} & \frac{1}{\sqrt[]{10}}
            \end{bmatrix} \begin{bmatrix}
            \sqrt[]{50} & 0 \\ 0 & 0
            \end{bmatrix}\begin{bmatrix}
            \sfrac{1}{\sqrt[]{5}} & \sfrac{2}{\sqrt[]{5}} \\ \sfrac{-2}{\sqrt[]{5}} & \sfrac{1}{\sqrt[]{5}}
            \end{bmatrix}$ \\
            Therefore $A^+=V\Sigma^+U^T=\begin{bmatrix}
            \sfrac{1}{\sqrt[]{5}} & \sfrac{-2}{\sqrt[]{5}} \\ \sfrac{2}{\sqrt[]{5}} & \sfrac{1}{\sqrt[]{5}}
            \end{bmatrix}\begin{bmatrix}
            \frac{1}{\sqrt[]{50}} & 0 \\ 0 & 0
            \end{bmatrix}\begin{bmatrix}
            \frac{1}{\sqrt[]{10}} & \frac{3}{\sqrt[]{10}} \\ \frac{-3}{\sqrt[]{10}} & \frac{1}{\sqrt[]{10}}
            \end{bmatrix}$
\item[b)] See attached paper.
\item[c)] See attached paper.
\item[d)] See attached paper.
\end{itemize}

\noindent \textbf{Problem S2}: This question focuses on the subspace 
$\mathcal{S} = \left\{ n\cos(x)+m\sin(x)\right\}$.  This subspace is composed of
all linear combinations of $\cos(x)$ and $\sin(x)$.  
\begin{itemize}
\item[(a)] Find a basis for $\mathcal{S}$.  (Show your work. Hint: This problem 
is similar to problems on $\mathcal{P}_2$.)
\item[(b)] Consider the derivative operator $d/dx$.  Determine the range and 
kernel for the derivative transformation when applied to $\mathcal{S}$. 
\item[(c)] Create the matrix representation for the derivative transformation 
when applied to $\mathcal{S}$.  
\item[(d)] Repeat parts (b) and (c) for the integral transformation $\int 
f(x)dx$ where the integration constant is always set to 0. \\
\end{itemize}

\textit{Solution.}
\begin{itemize}
\item[a)] A basis for $\mathcal{S}$ is 
			$\left\{\begin{bmatrix} \cos x \\ \sin x\end{bmatrix}\right\}$. It is the only vector, and so must be linearly independent.
            It is also a spanning set of $\mathcal{S}$ as stated in the prompt.
\item[b)] $T(v)=-n\sin x + m \cos x \Rightarrow$ range: $-n-m$ to $m+n$ inclusive. 
			kernel: $m=n=0$
\item[c)] $T=\begin{bmatrix}
			0 & -1 \\ 1 & 0
			\end{bmatrix}$
\item[d)] $T(v)=n\sin x + m \cos x \Rightarrow T=\begin{bmatrix}
0 & 1 \\ -1 & 0
\end{bmatrix}$ with the same range and kernel as the derivative transformation.
\end{itemize}

\noindent \textbf{Problem S3}: Consider the linear transformation $T: 
\mathcal{R}^4\rightarrow\mathcal{R}^3$ from the non-standard basis  $\left\{\vec{v}_1, \vec{v}_2, \vec{v}_3, \vec{v}_4\right\}$ of $\mathcal{R}^4$ to the 
non-standard basis $\left\{\vec{w}_1, \vec{w}_2, \vec{w}_3\right\}$ of 
$\mathcal{R}^3$ defined by $T(\vec{v}_1) = \sqrt{2}\vec{w}_1$, $T(\vec{v}_2) 
= \sqrt{8}\vec{w}_2$, $T(\vec{v}_3) = \vec{0}$, and $T(\vec{v}_4) = \vec{0}$.  
The non-standard bases are defined by \\
$\left\{\vec{v}_1, \vec{v}_2, \vec{v}_3, \vec{v}_4\right\}=\left\{ 
\left[\begin{array}{c}1/2\\ -1/2\\ 1/2\\ -1/2 
\end{array}\right],\left[\begin{array}{c}1/2\\ 1/2\\ 1/2\\ 1/2 
\end{array}\right],\left[\begin{array}{c} -1/\sqrt{2}\\ 
0 \\1/\sqrt{2}\\0\end{array}\right],\left[\begin{array}{c} 0\\1/\sqrt{2} \\ 0\\ 
-1\sqrt{2}\end{array}\right]  \right\}$ \\
$\left\{\vec{w}_1, \vec{w}_2, \vec{w}_3\right\}=\left\{ \left[\begin{array}{c}-
1/\sqrt{2}\\ 1/\sqrt{2}\\ 0\end{array}\right],\left[\begin{array}{c}1/\sqrt{2}\\
1/\sqrt{2}\\ 0 \end{array}\right],\left[\begin{array}{c} 0\\ 0\\ 
1\end{array}\right] \right\} $ 
\begin{itemize}
\item[(a)] What is the matrix representation of $T$.
\item[(b)] Using changes of bases matrices, construct a new linear 
transformation $S$ that converts vectors in the standard basis of 
$\mathcal{R}^4$ to the non-standard basis, applies $T$, and then converts the 
vectors to the standard basis of $\mathcal{R}^3$. Write the matrix 
representation of $S$ as a product of matrices. 
\item[(c)] What is the rank-1 approximation of the matrix representation of $S$?
\end{itemize}

\textit{Solution.}
\begin{itemize}
\item[a)] Based on the outputs of $T(v_i)$ with $1\leq i \leq 4$, $T=\begin{bmatrix}
			\sqrt[]{2} & 0 & 0 & 0 \\
            0 & \sqrt[]{8} & 0 & 0 \\
            0 & 0 & 0 & 0
			\end{bmatrix}$
\item[b)] Based on the shortcuts provided in class, $V$ be the change of basis matrix from standard basis to non-standard basis in $\mathcal{R}^4$ with rows made up of each $v$ vector. Let $W$ be the change of basis matrix from non-standard basis to standard basis in $\mathcal{R}^3$ with columns made up of the $w$ vectors. 
Then $S=WTV=\begin{bmatrix}
\vec{w}_1 & \vec{w}_2 & \vec{w}_3
\end{bmatrix}
T
\begin{bmatrix}
\vec{v}_1^T \\ \vec{v}_2^T \\ \vec{v}_3^T \\ \vec{v}_4^T
\end{bmatrix}$.
\item[c)] Notice that $S=WTV$ is the SVD of $S$. Therefore a rank one approximation of $S$ is 
$\begin{bmatrix}
\frac{1}{\sqrt[]{2}} \\ \frac{1}{\sqrt[]{2}} \\ 0
\end{bmatrix}
\sqrt[]{8}
\begin{bmatrix}
\frac{1}{2} & \frac{1}{2} & \frac{1}{2} & \frac{1}{2}
\end{bmatrix}=\begin{bmatrix}
1 & 1 & 1 & 1 \\
1 & 1 & 1 & 1 \\
0 & 0 & 0 & 0
\end{bmatrix}$
\end{itemize}

\end{document}
$\vec{u}=\left[\begin{array}{c} 1 \\ 0\end{array}\right]$
$\left[\begin{array}{cc}  & \\  & \end{array}\right]$
$\left[\begin{array}{ccc}  &  & \\  &  & \\ & & \end{array}\right]$