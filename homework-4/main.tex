\documentclass[12pt]{article}
\usepackage{amsmath,amsthm,amssymb,amsfonts,setspace, commath}

\textwidth 7.0 truein
\oddsidemargin -0.25in   %left-hand edge
\evensidemargin -0.5 truein  %right-hand edge
\topmargin -0.85in      %top of paper to top of head, pulls whole unit
\textheight 9.5in

%%%% In most cases you won't need to edit anything above this line %%%%

\begin{document}
\hfill Tanner Kvarfordt 

\hfill A02052217

\hfill Math 2270

\hfill Assignment \#4


\begin{itemize}
\item[2.7.3)] 
\begin{itemize}
\item[a)] The matrix $((AB)^{-1})^T$ comes from $(A^{-1})^T$ and $(B^{-1})^T$. In what order?
\item[b)] If $U$ is upper triangular then $(U^{-1})^T$ is $\rule{1cm}{0.15mm}$ triangular.
\end{itemize}

\textit{Solution.}
\begin{itemize}
\item[a)] $((AB)^{-1})^T = (B^{-1}A^{-1})^T = (A^{-1})^T(B^{-1})^T$
\item[b)] lower
\end{itemize}

\item[2.7.7)] True or false:
\begin{itemize}
\item[a)] The block matrix $\left[\begin{array}{cc} 0 & A \\ A & 0\end{array}\right]$ is automatically symmetric.
\item[b)] If $A$ and $B$ are symmetric, then their product, $AB$, is symmetric.
\item[c)] If $A$ is not symmetric, then $A^{-1}$ is not symmetric.
\item[d)] When $A$, $B$, $C$ are symmetric, the transpose of $ABC$ is $CBA$
\end{itemize}

\textit{Solution.}
\begin{itemize}
\item[a)] False. If $A$ is anything but a single entry (such as $A$=$\left[\begin{array}{cc} 1 & 3 \end{array}\right]$), then the block matrix is not square, and therefore cannot be symmetric.
\item[b)] False. Consider $\left[\begin{array}{cc} 0 & 2 \\ 2 & 1\end{array}\right]
\left[\begin{array}{cc} 1 & 3 \\ 3 & 1\end{array}\right]=\left[\begin{array}{cc} 6 & 2 \\ 5 & 7\end{array}\right]$
\item[c)] True. $A=A^T \Rightarrow A^TA^{-1}=I=A^T(A^T)^{-1} \Rightarrow (A^T)^{-1}=A^{-1}=(A^{-1})^T$.
\item[d)] True. $(ABC)^T=(BC)^TA^T=C^TB^TA^T$, and since $A$, $B$, $C$ are symmetric, $C^TB^TA^T=CBA$.
\end{itemize}
 

\item[2.7.39)] Suppose $Q^T$ \textit{equals} $Q^{-1}$ (transpose equals inverse, so $Q^TQ=I$)
\begin{itemize}
\item[a)] Show that columns $q_1,\ldots,q_n$ are unit vectors: $\norm{q_i}^2=1$
\item[b)] Show that every two columns of $Q$ are perpendicular: $q^T_1q_2=0$
\item[c)] Find a 2 by 2 example with first entry $q_{11}= \cos\theta$
\end{itemize}

\textit{Solution.}
\begin{itemize}
\item[a)] $(Q^TQ)_{ij}=q^T_i \cdot q_j$. When $i=j$, we are on the diagonal of $Q^TQ$, so $(Q^TQ)_{ij}=q_i \cdot q_i=1$. So $\norm{q_i}=1$
\item[b)] $(Q^TQ)_{ij}=q^T_i \cdot q_j$. When $i\neq j$, $(Q^TQ)_{ij}=q^T_i \cdot q_j=0$
\item[c)] $Q=Q^T=Q^{-1}=\left[\begin{array}{cc} \cos\theta & \sin\theta \\ \sin\theta & -\cos\theta\end{array}\right]$
\end{itemize}

\item[3.1.10)] Which of the following subsets of $R^3$ are actually subspaces?
\begin{itemize}
\item[a)] The plane of vectors $(b_1,b_2,b_3)$ with $b_1=b_2$
\item[b)] The plane of vectors with $b_1=1$
\item[c)] The vectors with $b_1b_2b_3=0$
\item[d)] All linear combinations of $\vec{u}=(1,4,0)$ and $\vec{t}=(2,2,2)$
\item[e)] All vectors that satisfy $b_1+b_2+b_3=0$
\item[f)] All vectors with $b_1 \leq b_2 \leq b_3$
\end{itemize}

\textit{Solution.}
\begin{itemize}
\item[a)] The vectors in this subset are of the form $(x, y, z)$ where $x=y$. Let $\vec{v}=(x_1,y_1,z_1)$ and $\vec{w}=(x_2,y_2,z_2)$, where $x_i=y_i$. $c\vec{v}=(x_1c,y_1c,z_1c)$ which fits the form for $x=x_1c$, $y=y_1c$, $z=z_1c$, and the $x$ and $y$ components still match since both were multiplied by the same constant, therefore $c\vec{v}$ is in the subset. $\vec{v}+\vec{w}=(x_1+x_2,y_1+y_2,z_1+z_2)$ which fits the form for $x=x_1+x_2$, $y=y_1+y_2$, $z=z_1c+z_2$, and the $x$ and $y$ components still match since they were both added to the same number, therefore $\vec{v}+\vec{w}$ is in the subset. Therefore, this subset is a subspace of $R^3$.
\item[b)] This subset is not a subspace of $R^3$ because it does not contain the zero vector.
\item[c)] The vectors in this subset are of the form $(x, y, z)$ where $xyz=0$. Let $\vec{v}=(x_1,y_1,z_1)$ and $\vec{w}=(x_2,y_2,z_2)$, where $x_iy_iz_i=0$. In order to meet the criteria that $xyz=0$, at least one component of every vector in the subset \textit{must} be 0. $\vec{v}+\vec{w}$ may result in all non-zero components ($\vec{v}=(0,1,6)$, $\vec{w}=(5,0,1)$ for example). Therefore this subset is not a subspace of $R^3$.
\item[d)] The vectors in this subset are of the form $a\vec{u}+b\vec{t}$. Let $\vec{v}=a_1\vec{u}+b_1\vec{t}$ and $\vec{w}=a_2\vec{u}+b_2\vec{t}$. 
\begin{equation*}
c\vec{v}=a_1c\vec{u}+b_1c\vec{t} \text{ which fits the form for } a=a_1c \text{, } b=b_1c 
\end{equation*}
\begin{equation*}
\therefore c\vec{v} \text{ is in the subset.}
\end{equation*}
\begin{equation*}
\vec{v}+\vec{w}=(a_1+a_2)\vec{u}+(b_1+b_2)\vec{t}\text{ which fits the form for } a=a_1+a_2 \text{, } b=b_1+b_2 
\end{equation*}
\begin{equation*}
\therefore \vec{v}+\vec{w} \text{ is in the subset.}
\end{equation*} Therefore, this subset is a subspace of $R^3$.
\item[e)] The vectors in this subset are of the form $(x, y, z)$ where $x+y+z=0$. Let $\vec{v}=(x_1,y_1,z_1)$ and $\vec{w}=(x_2,y_2,z_2)$, where $x_i+y_i+z_i=0$. $c\vec{v}=(x_1c,y_1c,z_1c)$ which fits the form for $x=x_1c$, $y=y_1c$, $z=z_1c$, and the sum of the components is still 0 due to the distributive property of addition, therefore $c\vec{v}$ is in the subset. $\vec{v}+\vec{w}=(x_1+x_2,y_1+y_2,z_1+z_2)$ which fits the form for $x=x_1+x_2$, $y=y_1+y_2$, $z=z_1c+z_2$, and the sum of the components is still 0 due to the commutative property of addition, therefore $\vec{v}+\vec{w}$ is in the subset. Therefore this subset is a subspace of $R^3$.
\item[f)] The vectors in this subset are of the form $(x, y, z)$ where $x \leq y \leq z$. Let $\vec{v}=(x_1,y_1,z_1)$ and $\vec{w}=(x_2,y_2,z_2)$, where $x_i \leq y_i \leq z_i$. $c\vec{v}=(x_1c,y_1c,z_1c)$ which fits the form for $x=x_1c$, $y=y_1c$, $z=z_1c$, therefore $c\vec{v}$ is in the subset. $\vec{v}+\vec{w}=(x_1+x_2,y_1+y_2,z_1+z_2)$ which fits the form for $x=x_1+x_2$, $y=y_1+y_2$, $z=z_1+z_2$, therefore $\vec{v}+\vec{w}$ is in the subset. Therefore, this subset is a subspace of $R^3$.
\end{itemize}

\item[3.1.17)]
\begin{itemize}
\item[a)] Show that the set of \textit{invertible} matrices in $\mathbf{M}$ is not a subspace. 
\item[b)] Show that the set of \textit{singular} matrices in $\mathbf{M}$ is not a subspace.
\end{itemize}

\textit{Solution.}
\begin{itemize}
\item[a)] The set of invertible matrices does not contain the zero matrix, and therefore is not a subspace.
\item[b)] Consider the singular 3 by 3 matrices $A=\left[\begin{array}{ccc} 1 & 0 & 0 \\ 0 & 0 & 0 \\ 0 & 0 & 0\end{array}\right]$ and $B=\left[\begin{array}{ccc} 0 & 0 & 0 \\ 0 & 1 & 0 \\ 0 & 0 & 1\end{array}\right]$. $A+B=I$, which is not singular, and therefore $A+B$ does not fit the form of the subset, and therefore the subset cannot be a subspace.
\end{itemize}

\item[3.1.27)] True or false (with a counterexample if false):
\begin{itemize}
\item[a)] The vectors $b$ that are not in the column space $C(A)$ form a subspace.
\item[b)] If $C(A)$ contains only the zero vector, then $A$ is the zero matrix.
\item[c)] The column space of $2A$ equals the column space of $A$.
\item[d)] The column space of $A-I$ equals the column space of $A$ (test this).
\end{itemize}
\textit{Solution.}
\begin{itemize}
\item[a)] False. Let $A=\left[\begin{array}{ccc} 1 & 0 \\ 0 & 0 \end{array}\right]$. Then $C(A)=\text{span}((x_1,0))$. Neither of the two vectors $(1,1)$, or $(0, -1)$ are in the column space of $A$, but their sum, $(1,0)$ is in the column space of $A$, thus failing the subspace test.
\item[b)] True. $C(A)$ is the set of all linear combinations of the columns of $A$. If the only linear combination of $A$'s columns that exists is the zero vector, then $A$ must be the zero matrix (since all columns of $A$ are also present in $C(A)$). 
\item[c)] True. $C(A)=\text{span}(\vec{a_1},\ldots,\vec{a_n})$ and $C(2A)=\text{span}(2\vec{a_1},\ldots,2\vec{a_n})$. Since $\vec{a_1},\ldots,\vec{a_n}$ and $2\vec{a_1},\ldots,2\vec{a_n}$ are linearly dependent of each other (consider $c_1(\vec{a_1}+\ldots+\vec{a_n})+c_2(2\vec{a_1}+\ldots+2\vec{a_n})=0$ where $c_1=-2$ and $c_2=1$), \[\text{span}(\vec{a_1},\ldots,\vec{a_n})=\text{span}(2\vec{a_1},\ldots,2\vec{a_n})=C(A)=C(2A)\]
\item[d)] False. Consider $A=\left[\begin{array}{cc} 1 & 0 \\ 0 & 1\end{array}\right]$. Then $C(A)=\text{span}((1,0),(0,1))=\mathbb{R}^2$.\[A-I=\left[\begin{array}{cc} 1 & 0 \\ 0 & 1\end{array}\right]-\left[\begin{array}{cc} 1 & 0 \\ 0 & 1\end{array}\right]=\left[\begin{array}{cc} 0 & 0 \\ 0 & 0\end{array}\right]\] and $C(A-I)=\text{span}(\vec{0})\neq\mathbb{R}^2$.
\end{itemize}

\item[3.1.29)] If the 9 by 12 system $Ax=b$ is solvable for every $b$, then $C(A)=$ $\rule{1cm}{0.15mm}$.

\textit{Solution.} $\mathbb{R}^9$

\item[S1)] The eight conditions that define a vector space are on page 131 of the textbook.  Suppose multiplication of a vector $\vec{x} = (x_1,x_2)$ by $c$, yielding $c\vec{x}$, is defined to produce $(cx_1,0)$ instead of $(cx_1,cx_2)$.
\begin{itemize}
\item[a)]  Which of the eight conditions for a vector space are not satisfied?
\item[b)] Find a subspace of $\mathbf{R}^2$ where the above definition for multiplication does satisfy the eight conditions.
\end{itemize}
\textit{Solution.}
\begin{itemize}
\item[a)] Condition 5 is not satisfied. $1\vec{x}=1(x_1,x_2)=(1x_1,0)\neq\vec{x}$
\item[b)] The subspace containing only the zero vector.
\end{itemize}

\item[S2)]
\begin{itemize}
\item[a)] When is a triangular $n$-by-$n$ matrix invertible?  Be sure to justify your answer.
\item[b)] What is $(\mathbf{A}\mathbf{B})^2$? Why do we know that $\mathbf{A}^2\mathbf{B}^2 = (\mathbf{AB})^2$ may not be true?
\end{itemize}
\textit{Solution.}
\begin{itemize}
\item[a)] When it has $n$ pivots/when its columns are linearly independent/when its diagonal entries are all non-zero. If it has non-zero entries in the diagonals as a triangular matrix, it is guaranteed to have $n$ pivots.
\item[b)] $(AB)^2=ABAB$ whereas $A^2B^2=AABB$. Order matters in matrix multiplication, and so the two are not necessarily equivalent.
\end{itemize}
\end{itemize}


\end{document}


\end{document}


$\vec{u}=\left[\begin{array}{c} 1 \\ 0\end{array}\right]$

$\left[\begin{array}{cc}  & \\  & \end{array}\right]$

$\left[\begin{array}{ccc}  &  & \\  &  & \\ & & \end{array}\right]$


CREATE VECTOR
$\vec{u}=\left[\begin{array}{c} 1 \\ 0\end{array}\right]$

CREATE EQUATION
$A = \begin{align}{cc}
 1 & 2\\
3 & 4
\end{align}$

CREATE EQUATION
\begin{equation}
y=x
\end{equation}

CREATE SYSTEM OF EQUATIONS
\begin{align}
y&=x\\
y&=x
\end{equation}

